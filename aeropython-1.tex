% Created 2021-01-31 dom 19:41
% Intended LaTeX compiler: pdflatex
\documentclass[11pt]{article}
\usepackage[utf8]{inputenc}
\usepackage[T1]{fontenc}
\usepackage{graphicx}
\usepackage{grffile}
\usepackage{longtable}
\usepackage{wrapfig}
\usepackage{rotating}
\usepackage[normalem]{ulem}
\usepackage{amsmath}
\usepackage{textcomp}
\usepackage{amssymb}
\usepackage{capt-of}
\usepackage{hyperref}
\author{Marco Callari}
\date{\today}
\title{Appunti di Aerodinamica}
\hypersetup{
 pdfauthor={Marco Callari},
 pdftitle={Appunti di Aerodinamica},
 pdfkeywords={},
 pdfsubject={},
 pdfcreator={Emacs 27.1 (Org mode 9.4)}, 
 pdflang={English}}
\begin{document}

\maketitle
\tableofcontents


\section{Concetti base}
\label{sec:org8350f1a}
\subsection{Teorema del trasporto di Reynolds}
\label{sec:org0c62930}
Data:
\begin{itemize}
\item Una funzione \(f\) che dipende da spazio e tempo \(f(x,t)\)
\end{itemize}
Di cui voglio calcolare la derivata nel tempo dell'integrale nel volume di controllo \(\Omega(t)\) (nota: variabile nel tempo), ovvero:
\begin{gather*}
\frac{d}{dt} \int_{\Omega(t)} f \,dV
\end{gather*}
Utilizzando il teorema di Reynolds, è possibile riscrivere questa derivata di un integrale come:
\begin{gather*}
\frac{d}{dt} \int_{\Omega(t)} f \,dV = \int_{\Omega(t)}{\frac{\partial f}{\partial t}\,dV} + \int_{\partial\Omega(t)}{(v^b \cdot n)f\,dA}
\end{gather*}
Dove:
\begin{itemize}
\item \(\partial \Omega(t)\) è la superficie esterna del volume di controllo (superficie di controllo)
\item \(n\) è il versore normale alla superficie \(\delta \Omega(t)\)
\item \(v^b(x,t)\) è la velocità dell'elemento di area (@TODO: specificare)
\end{itemize}
\end{document}
