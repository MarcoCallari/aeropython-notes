% Created 2021-04-17 sab 11:46
% Intended LaTeX compiler: pdflatex
\documentclass[11pt]{article}
\usepackage[utf8]{inputenc}
\usepackage[T1]{fontenc}
\usepackage{graphicx}
\usepackage{grffile}
\usepackage{longtable}
\usepackage{wrapfig}
\usepackage{rotating}
\usepackage[normalem]{ulem}
\usepackage{amsmath}
\usepackage{textcomp}
\usepackage{amssymb}
\usepackage{capt-of}
\usepackage{hyperref}
\author{Marco Callari}
\date{\today}
\title{Appunti di Aerodinamica}
\hypersetup{
 pdfauthor={Marco Callari},
 pdftitle={Appunti di Aerodinamica},
 pdfkeywords={},
 pdfsubject={},
 pdfcreator={Emacs 27.2 (Org mode 9.5)}, 
 pdflang={English}}
\begin{document}

\maketitle
\tableofcontents


\section{Concetti base}
\label{sec:orgbc00752}
\subsection{Punto di vista Lagrangiano}
\label{sec:orgb277e85}
Seguo un volume di controllo nello spazio e ne descrivo l'evoluzione
\subsection{Punto di vista Euleriano}
\label{sec:org1951e8e}
Fisso gli occhi su un punto in specifico e ne descrivo l'evoluzione
\subsection{Differenziale di una funzione}
\label{sec:org0ca07f4}
Data una funzione \(f\) = \(f(x)\) il suo differenziale \(df\) è definito come:
\begin{gather*}
df(x,\Delta x) = \frac{df}{dx} \Delta x
\end{gather*}
\subsection{Derivazione derivata materiale}
\label{sec:org91d4ca6}
Usata per esprimere la variazione di una proprietà del fluido da un punto di vista Lagrangiano (ovvero seguendo  l'evoluzione di un volume infitesimo di fluido)
Definita come:
\begin{gather*}
\frac{Df}{Dx} \equiv \lim_{x_2 \rightarrow x_1} \frac{f(x_2) - f(x_1)}{x_2 - x_1}
\end{gather*}
Data una funzione \(f = f(x,y,z,t)\), posso approssimare il suo valore in un punto \(P_2\) \(\rightarrow\) \(P_1\) all'istante \(t_2\) \(\rightarrow\) \(t_1\) tramite una serie di Taylor al primo ordine:
\begin{gather*}
f(x_2,y_2,z_2,t_2) = f(x_1,y_1,z_1,t_1) + \frac{\partial f}{\partial x} (x_2-x_1) + \frac{\partial f}{\partial y} (y_2 - y_1) + \frac{\partial f}{\partial z} (z_2 - z_1) + \frac{\partial f}{\partial t} (t_2 - t_1)
\end{gather*}
Dividendo per \((t_2 - t_1)\) :
\begin{gather*}
\frac{f(x_2,y_2,z_2,t_2) - f(x_1,y_1,z_1,t_1)}{t_2 - t_1} = \frac{\partial f}{\partial x} u + \frac{\partial f}{\partial y} v + \frac{\partial f}{\partial z} w + \frac{\partial f}{\partial t}
\end{gather*}
Ovvero:
\begin{gather*}
\frac{Df}{Dt} = \frac{\partial f}{\partial t} + \frac{\partial f}{\partial x} u + \frac{\partial f}{\partial y} v + \frac{\partial f}{\partial z} w
\end{gather*}
Dove \(u\), \(v\), \(w\), sono le velocità in direzione \(x\), \(y\), \(z\)
Può essere espressa anche come:
\begin{gather*}
\frac{Df}{Dt} = \frac{\partial f}{\partial t} + \vec{v} \cdot \nabla f
\end{gather*}

\subsection{Teorema del trasporto di Reynolds}
\label{sec:org0500936}
Data:
\begin{itemize}
\item Una funzione \(f\) che dipende da spazio e tempo \(f(x,t)\)
\end{itemize}
Di cui voglio calcolare la derivata nel tempo dell'integrale nel volume di controllo \(\Omega(t)\) (nota: variabile nel tempo), ovvero:
\begin{gather*}
\frac{d}{dt} \int_{\Omega(t)} f \,dV
\end{gather*}
Utilizzando il teorema di Reynolds, è possibile riscrivere questa derivata di un integrale come:
\begin{gather*}
\frac{d}{dt} \int_{\Omega(t)} f \,dV = \int_{\Omega(t)}{\frac{\partial f}{\partial t}\,dV} + \int_{\partial\Omega(t)}{(v^b \cdot n)f\,dA}
\end{gather*}
Dove:
\begin{itemize}
\item \(\partial \Omega(t)\) è la superficie esterna del volume di controllo (superficie di controllo)
\item \(n\) è il versore normale alla superficie \(\delta \Omega(t)\)
\item \(v^b(x,t)\) è la velocità dell'elemento di area
\end{itemize}

\subsection{Equazione di conservazione della massa}
\label{sec:org20cd8c9}
Derivata dal un punto di vista EULERIANO. Considero un volume infinitesimo che circonda un punto \(x\):
@TODO aggiungere immagine
Definisco \(\dot{m}(x)\) portata di fluido nel punto \(x\). Usando una serie di Taylor, posso descrivere la portata uscente dalla parete di destra del volume di controllo come:
\begin{gather*}
\dot{m}(x+\frac{dx}{2}) =\dot{m}(x) + \frac{\partial \dot{m}}{\partial x} \frac{dx}{2}
\end{gather*}
In maniera analoga, la massa uscente dalla parete sinistra del volume di controllo:
\begin{gather*}
\dot{m}(x-\frac{dx}{2}) =\dot{m}(x) - \frac{\partial \dot{m}}{\partial x} \frac{dx}{2}
\end{gather*}
Prendendo in considerazione esclusivamente l'asse \(x\), la variazione di massa per unità di tempo risulta essere uguale a:
\begin{gather*}
\dot{m}(x+\frac{dx}{2}) - \dot{m}(x-\frac{dx}{2}) =\frac{\partial \dot{m}}{\partial x} dx
\end{gather*}
Dato che \(\dot{m} = \rho u dy dz\)
\begin{gather*}
\dot{m}(x+\frac{dx}{2}) - \dot{m}(x-\frac{dx}{2}) =\frac{\partial (\rho u)}{\partial x} dx dy dz
\end{gather*}
Eseguendo in maniera analoga gli stessi calcoli anche per l'asse y e l'asse z, la variazione totale di massa per unità di tempo nel volume di controllo risulta essere:
\begin{gather*}
(\frac{\partial (\rho u)}{\partial x} + \frac{\partial (\rho v)}{\partial y} + \frac{\partial (\rho w)}{\partial z}) dx dy dz
\end{gather*}
Visto che stiamo analizzando il volume di controllo da un punto di vista Euleriano, la variazione di massa può essere descritta come:
\begin{gather*}
\frac{dm}{dt} = \frac{d \rho}{dt} dx dy dz
\end{gather*}
Combinando le due equazioni:
\begin{gather*}
- \frac{d \rho}{dt} = \frac{\partial (\rho u)}{\partial x} + \frac{\partial (\rho v)}{\partial y} + \frac{\partial (\rho w)}{\partial z}
\end{gather*}
(Aggiunto un - in quanto il primo termine è negativo quando la massa del volume di controllo diminusce, mentre il secondo, in questo caso, è positivo)
Concludendo,
\begin{gather*}
\frac{d \rho}{dt} + \nabla \cdot (\rho \vector{v})= 0
\end{gather*}
\subsection{Equazione di conservazione della quantità di moto}
\label{sec:orgc8c24eb}
@TODO
\subsection{Fluido incomprimibile \(\rightarrow\) \(\nabla \cdot \vector{v}\) = 0}
\label{sec:org58aab6e}
Data l'equazione di conservazione della massa,
\begin{gather*}
\frac{d \rho}{dt} + \nabla \cdot (\rho \vector{v})= 0
\end{gather*}
Imponendo \(\rho\) = const., ricavo che:
\begin{gather*}
\rho \nabla \cdot \vector{v}= 0
\end{gather*}
Ergo,
\begin{gather*}
\nabla \cdot \vector{v}= 0
\end{gather*}
\subsection{Relazione tra \(\Delta\) p \(\rightarrow\) \(\Delta\) v}
\label{sec:orge4aab30}
@TODO
\subsection{Pathline, streamline, timeline, streakline}
\label{sec:org5a77504}
\begin{itemize}
\item Pathline
\end{itemize}
Seguo il movimento di un elemento di fluido da un istante \(t_0\) ad un istante \(t_0\) + \(\Delta\) \(t\). Traccio la traiettoria di questo elemento, ottenendo appunto una pathline.
\begin{itemize}
\item Streakline
\end{itemize}
Prendo un punto \(\vec{x_0}\) nel piano ad un istante \(t_0\). Traccio la traiettoria che unisce tutti gli elementi di fluido che sono passati per quel punto durante un lasso di tempo \(\Delta\) \(t\), ottenendo appunto una streakline.
\begin{itemize}
\item Timeline
\end{itemize}
Dati n elementi normali alla velocità del fluido rilasciati nello stesso istante \(t_0\), evidenzio la loro posizione ad un istante \(t_1\). Utile per visualizzare il profilo di velocità attorno ad un profilo alare.
\begin{itemize}
\item Streamline
\end{itemize}
Curva tangente in ogni punto alla velocità del fluido.
\subsection{Equazione streamline}
\label{sec:orgd463c26}
Visto che la streamline deve essere tangente alla velocità in ogni punto, se prendo un segmento infinitesimo \(dS\) della streamline \(S\) posso dedurre geometricamente che
\begin{gather*}
    \frac{dY}{dX} = \frac{V}{U}
\end{gather*}
\begin{gather*}
    \frac{dY}{V} = \frac{dX}{U}
\end{gather*}
\subsection{Potenziale della velocità}
\label{sec:orgb614d2d}
Funzione \(\phi\) definita come:
    \begin{gather*}
\frac{\partial \phi}{\partial x} = v_x , \frac{\partial \phi}{\partial y} = v_y
    \end{gather*}
\subsection{Streamfunction}
\label{sec:org301627d}
Funzione matematica \(\psi\) di ordine superiore alla velocità utilizzata per descrivere e visualizzare il flusso. Definita come:
\begin{gather*}
    \frac{d \psi}{dy} = u
\end{gather*}
\begin{gather*}
    \frac{d \psi}{dx} = -v
\end{gather*}
\(\psi\) è costante lungo una streamline. @TODO: spiegare meglio ed aggiungere forma polare.
\section{Flusso potenziale}
\label{sec:orgf026e3e}
\subsection{Assiomi fondamentali}
\label{sec:org9d5959c}
Proprietà fluido:
\begin{itemize}
\item Inviscido
\item Incomprimibile
\item Subsonico
\item Stazionario
\item Campo di velocità irrotazionale
\end{itemize}
Date queste ipotesti, le equazioni governanti il fluido risultano \uline{lineari}. Grazie a questa proprietà, è possibile sovrapporre due soluzioni note per ricavare il risultato di una terza.
\subsection{Circolazione}
\label{sec:org3a2b329}
\begin{gather*}
    \Gamma = \oint \mathbf v \cdot \vec{dl}
\end{gather*}
Ovvero, la circolazione è l'integrale curvilineo lungo un percorso chiuso.
\subsection{Teorema di Stokes}
\label{sec:org9e3c350}
Il flusso del rotore di un campo vettoriale \(F\) attraverso una superficie \(S\) è equivalente all'integrale di linea sul bordo della superficie \(\Gamma\)
\begin{gather*}
\oint_{\Gamma} \mathbf{F} \cdot \mathrm{d} \Gamma=\iint_{S}(\nabla \times \mathbf{F}) \cdot \hat{\mathbf{n}} d S
\end{gather*}
\subsection{Equazione di Laplace}
\label{sec:orgf4f368f}
Partendo dall'equazione della circolazione:
\begin{gather*}
    \Gamma = \oint \mathbf v \cdot \vec{dl}
\end{gather*}
è possibile applicare il teorema di Stokes per ottenere una forma equivalente di questo integrale
\begin{gather*}
\oint_{l} \mathbf{v} \cdot d \vec{l}=\iint_{s} (\nabla \times \mathbf v) d s = \iint_{s} \omega \cdot \vec{n} d s
\end{gather*}
Dato che il campo vettoriale della velocità è irrotazionale, \(\nabla\) \(\times\) \(v\) = 0 ( e ovviamente \(\omega\) = 0 ). Possiamo quindi affermare che :
\begin{gather*}
\oint_{l} \mathbf{v} \cdot d \vec{l} = 0
\end{gather*}
Quest'ultima equazione stipula l'esistenza di un campo vettoriale conservativo \(\phi\) tale che \(\mathbf v\) = \(\nabla \phi\).
Sapendo anche che in un flusso incomprimibile \newline \(\nabla\) \(\cdot\) \(\mathbf\) \(v\) = 0, possiamo ricavare l'equazione di Laplace:
\begin{gather*}
    \nabla ^2 \phi = 0
\end{gather*}
\subsection{Source (sorgente)}
\label{sec:org354ade4}
\subsubsection{Definizione}
\label{sec:orgcf66bd0}
Punto nello spazio che emette una portata \(\sigma\).
\subsubsection{Distribuzione velocità flusso}
\label{sec:org6b2dae1}
Il flusso ha esclusivamente una componente radiale. È possibile ottenere una forma analitica della velocità in funzione della portata emessa dal punto. Considerando il caso 2D ed applicando il principio di conservazione della massa, ricavo un integrale di linea attorno alla sorgente:
\begin{gather*}
    \sigma = \int_{0}^{2\pi}{\rho u dl}
\end{gather*}
Sapendo che la velocità ha soltanto una componente radiale (per ipotesi),
\begin{gather*}
    \sigma = 2 \pi \rho u_r
\end{gather*}
\begin{gather*}
    u_r = \frac{\sigma}{2 \pi \rho}
\end{gather*}
\subsubsection{Potenziale della velocità}
\label{sec:org7cd008b}
Ricordando la definizione di potenziale della velocità  (vedi capitolo \ref{sec:orgb614d2d})
\subsubsection{Streamline}
\label{sec:orgce9c3e3}
\subsection{Sink (pozzo)}
\label{sec:org8e94455}
\subsubsection{Definizione}
\label{sec:org3c0e2d0}
Punto nello spazio in cui converge una portata \(\sigma\) (negativa).
\subsubsection{Distribuzione velocità flusso}
\label{sec:orgdabaff9}
Il flusso ha esclusivamente una componente radiale. È possibile ottenere una forma analitica di questa componente radiale in funzione della portata emessa dal punto.
\subsubsection{Potenziale}
\label{sec:org1f0e7f4}
\subsubsection{Streamline}
\label{sec:org57a8b81}
\subsection{Doublet}
\label{sec:orgc12bb1e}
\end{document}
